\documentclass[a4paper,12pt,titlepage]{article}
\usepackage{amsmath} 
\usepackage{amssymb}
\usepackage[nottoc]{tocbibind}
\usepackage{float}
\usepackage{indentfirst}
\author{\textit{Jiang Yicheng}\\\textit{515370910224}}
\title{\textbf{VV286\\ Honors Mathematics IV\\
Ordinary Differential Equations\\
		Assignment 8}}
\date{\today}
\usepackage{extarrows}
\usepackage{mathrsfs}
\usepackage{dsfont}
\usepackage[top=0.8in, bottom=0.8 in, left= 0.8in, right=0.8 in]{geometry}
\usepackage{fancyhdr,lastpage}
	\pagestyle{fancy}
	\fancyhf{}
\cfoot{Page \thepage\ of \pageref{LastPage}}
\usepackage{multirow}
\usepackage{gauss}
\usepackage{geometry}
\usepackage{graphicx}
\begin{document}

\maketitle

\section*{Exercise 8.2}
\subsection*{i)}
Set $x(t)=\sum\limits_{k=0}^{\infty}a_kt^k$, then
$$tx'(t)=t\sum\limits_{k=1}^{\infty}ka_kt^{k-1}=\sum\limits_{k=0}^{\infty}(k+1)a_{k+1}t^{k+1}$$
$$x''(t)=\sum\limits_{k=2}^{\infty}k(k-1)a_kt^{k-2}=\sum\limits_{k=0}^{\infty}(k+2)(k+1)a_{k+2}t^{k}$$
Since 
$$x''-2tx'+\lambda x=0$$
then we can obtain that
$$\sum\limits_{k=0}^{\infty}(k+2)(k+1)a_{k+2}t^{k}-2\sum\limits_{k=0}^{\infty}(k+1)a_{k+1}t^{k+1}+\lambda\sum\limits_{k=0}^{\infty}a_kt^k=0$$
i.e.
$$(2a_2+\lambda a_0)+\sum\limits_{k=0}^{\infty}((k+3)(k+2)a_{k+3}-2(k+1)a_{k+1}+\lambda a_{k+1})t^{k+1}=0$$
So
$$\left\{
\begin{aligned}
&2a_2+\lambda a_0=0\\
&(k+3)(k+2)a_{k+3}+(\lambda-2(k+1))a_{k+1}=0,k\in\mathbb{N}\\
\end{aligned}
\right.$$
So for $k=2j+1,j\geqslant1$
$$a_{2j+1}=\dfrac{4j-2-\lambda}{(2j+1)(2j)}a_{2j-1}=\dfrac{\prod\limits_{i=1}^j(4i-2-\lambda)}{(2j+1)!}a_1$$
for $k=2j+2,j\geqslant0$
$$a_2=\dfrac{-\lambda}{2}a_0=\dfrac{0-\lambda}{2}a_0$$
$$a_{2j+2}=\dfrac{4j-\lambda}{(2j+2)(2j+1)}a_{2j}=\dfrac{\prod\limits_{i=1}^j(4i-\lambda)}{(2j+2)!/2}a_2=\dfrac{\prod\limits_{i=0}^j(4i-\lambda)}{(2j+2)!}a_0\,\,\,\,(j\geqslant1)$$
Set $a_0=c_1,a_1=0$ and we get
$$x_1(t)=c_1+\sum\limits_{j=0}^{\infty}\dfrac{\prod\limits_{i=0}^j(4i-\lambda)}{(2j+2)!}c_1t^{2j}$$
Set $a_0=0,a_1=c_2$ and we get
$$x_2(t)=c_2t+\sum\limits_{j=1}^{\infty}\dfrac{\prod\limits_{i=1}^j(4i-2-\lambda)}{(2j+1)!}c_2t^{2j+1}$$

\subsection*{ii)}
If $n=2m+1,m\in\mathbb{N}$, then $\forall j\geqslant m+1$, since $\lambda=2n=4m+2$
$$a_{2j+1}=\dfrac{\prod\limits_{i=1}^j(4i-2-\lambda)}{(2j+1)!}a_1=0$$
and $\forall j\leqslant m$
$$a_{2j+1}=\dfrac{\prod\limits_{i=1}^j(4i-2-\lambda)}{(2j+1)!}a_1\neq0$$
and therefore if $m>0$
$$x_2(t)=c_2t+\sum\limits_{j=1}^{\infty}\dfrac{\prod\limits_{i=1}^j(4i-2-\lambda)}{(2j+1)!}c_2t^{2j+1}=c_2t+\sum\limits_{j=1}^{m}\dfrac{\prod\limits_{i=1}^j(4i-2-\lambda)}{(2j+1)!}c_2t^{2j+1}$$
if $m=0$, $x_2(t)=c_2t$

This is a polynomial of degree $n$ and it is a solution to Hermite equation.

If $n=2m,m\in\mathbb{N}$, then $\forall j\geqslant m+1$, since $\lambda=2n=4m$
$$a_{2j+2}=\dfrac{\prod\limits_{i=0}^j(4i-\lambda)}{(2j+2)!}a_0=0$$
and $\forall j\leqslant m$
$$a_{2j+2}=\dfrac{\prod\limits_{i=0}^j(4i-\lambda)}{(2j+2)!}a_0\neq0$$
and therefore if $m>0$
$$x_1(t)=c_1+\sum\limits_{j=0}^{\infty}\dfrac{\prod\limits_{i=0}^j(4i-\lambda)}{(2j+2)!}c_1t^{2j}=c_1+\sum\limits_{j=0}^{m}\dfrac{\prod\limits_{i=0}^j(4i-\lambda)}{(2j+2)!}c_1t^{2j}$$
if $m=0$, $x_1(t)=c_1$
This is a polynomial of degree $n$ and it is a solution to Hermite equation.

To sum up, the Hermite equation has a polynomial solution of degree $n$ if $\lambda = 2n$.
\section*{Exercise 8.3}
\subsection*{i)}
Set $y(t)=t^r\sum\limits_{k=0}^{\infty}a_kt^k$, then
$$y'(t)=\sum\limits_{k=0}^{\infty}(k+r)a_kt^{k+r-1}$$
$$y''(t)=\sum\limits_{k=0}^{\infty}(k+r)(k+r-1)a_kt^{k+r-2}$$
Since 
$$2ty''+(1-2t)y'-y=0$$
then we can obtain that
\begin{align*}
0=&2t\sum\limits_{k=0}^{\infty}(k+r)(k+r-1)a_{k}t^{k+r-2}+(1-2t)\sum\limits_{k=0}^{\infty}(k+r)a_kt^{k+r-1}-\sum\limits_{k=0}^{\infty}a_kt^{k+r}\\
=&\sum\limits_{k=0}^{\infty}2(k+r)(k+r-1)a_{k}t^{k+r-1}+\sum\limits_{k=0}^{\infty}(k+r)a_{k}t^{k+r-1}-\sum\limits_{k=0}^{\infty}2(k+r)a_k t^{k+r}-\sum\limits_{k=0}^{\infty}a_k t^{k+r}\\
=&(2r(r-1)a_0+ra_0)t^{r-1}\\
&+\sum\limits_{k=0}^{\infty}(2(k+r+1)(k+r)a_{k+1}+(k+r+1)a_{k+1}-2(k+r)a_k-a_k)t^{k+r}
\end{align*}

So
$$\left\{
\begin{aligned}
&2r(r-1)a_0+ra_0=0\\
&(k+r+1)(2k+2r+1)a_{k+1}=(2k+2r+1)a_k,k\in\mathbb{N}^*\\
\end{aligned}
\right.\Leftrightarrow \left\{
\begin{aligned}
&r=0\vee r=\frac{1}{2}\\
&a_{k+1}=\dfrac{1}{k+r+1}a_k,k\in\mathbb{N}\\
\end{aligned}
\right.$$
then for $r=0$ we obtain that $\forall k\in\mathbb{N}, a_k=\dfrac{1}{k!}a_0$
$$x_1(t)=\sum\limits_{k=0}^{\infty}a_kt^k=a_0\sum\limits_{k=0}^{\infty}\dfrac{1}{k!}t^k=a_0e^t$$
for $r=\dfrac{1}{2}$ we obtain that $\forall k\in\mathbb{N}^*, a_k=a_0\prod\limits_{i=1}^k\dfrac{1}{i+\frac{1}{2}}=a_0\dfrac{2^k}{\prod\limits_{i=1}^k(2i+1)}=a_0\dfrac{2^k}{\prod\limits_{i=0}^k(2i+1)}$
$$x_2(t)=t^{\frac{1}{2}}\sum\limits_{k=0}^{\infty}a_kt^k=a_0t^{\frac{1}{2}}\sum\limits_{k=0}^{\infty}\dfrac{2^k}{\prod\limits_{i=0}^k(2i+1)}t^k$$

So two independent solutions of $2ty''+(1-2t)y'-y=0$ are
$$y_1(t)=c_1e^t,y_2(t)=c_2t^{\frac{1}{2}}\sum\limits_{k=0}^{\infty}\dfrac{2^k}{\prod\limits_{i=0}^k(2i+1)}t^k,c_1,c_2\in\mathbb{R}$$

\subsection*{ii)}
Set $y(t)=t^r\sum\limits_{k=0}^{\infty}a_kt^k$, then
$$y'(t)=\sum\limits_{k=0}^{\infty}(k+r)a_kt^{k+r-1}$$
$$y''(t)=\sum\limits_{k=0}^{\infty}(k+r)(k+r-1)a_kt^{k+r-2}$$
Since 
$$t^2y''+(t-t^2)y'-y=0$$
then we can obtain that
\begin{align*}
0=&t^2\sum\limits_{k=0}^{\infty}(k+r)(k+r-1)a_{k}t^{k+r-2}+(t-t^2)\sum\limits_{k=0}^{\infty}(k+r)a_kt^{k+r-1}-\sum\limits_{k=0}^{\infty}a_kt^{k+r}\\
=&\sum\limits_{k=0}^{\infty}(k+r)(k+r-1)a_{k}t^{k+r}+\sum\limits_{k=0}^{\infty}(k+r)a_{k}t^{k+r}-\sum\limits_{k=0}^{\infty}(k+r)a_k t^{k+r+1}-\sum\limits_{k=0}^{\infty}a_k t^{k+r}\\
=&(r(r-1)a_0+ra_0-a_0)t^{r}\\
&+\sum\limits_{k=1}^{\infty}((k+r)(k+r-1)a_{k}+(k+r)a_{k}-(k+r-1)a_{k-1}-a_k)t^{k+r}
\end{align*}

So
$$\left\{
\begin{aligned}
&r(r-1)a_0+ra_0-a_0=0\\
&(k+r+1)(k+r-1)a_{k}=(k+r-1)a_{k-1},k\in\mathbb{N}^*\\
\end{aligned}
\right.$$
then for $r=1$ we obtain that $\forall k\in\mathbb{N}, a_k=\dfrac{2}{(k+2)!}a_0$
$$x_1(t)=\sum\limits_{k=0}^{\infty}a_kt^{k+1}=a_0\sum\limits_{k=0}^{\infty}\dfrac{2}{(k+2)!}t^{k+1}=-a_0(\dfrac{1}{t}+1)+a_0\dfrac{e^t}{t}$$
for $r=-1$ we obtain that $\forall k\in\mathbb{N}^*,k\geqslant3, a_k=a_2\prod\limits_{i=3}^k\dfrac{1}{i}=\dfrac{2a_2}{k!},a_1=a_0$, 
$$x_2(t)=\sum\limits_{k=0}^{\infty}a_kt^{k-1}=\dfrac{a_0}{t}+a_0+2a_2\sum\limits_{k=0}^{\infty}\dfrac{1}{k!}t^{k-1}-2a_2(\dfrac{1}{t}+1)=(a_0-2a_2)(\dfrac{1}{t}+1)+2a_2\dfrac{e^t}{t}$$

So two independent solutions for $t^2y''+(t-t^2)y'-y=0$ is
$$y_1(t)=c_1(\dfrac{1}{t}+1),y_2(t)=c_2e^t,c_1,c_2\in\mathbb{R}$$
\subsection*{iii)}
Set $y(t)=t^r\sum\limits_{k=0}^{\infty}a_kt^k$, then
$$y'(t)=\sum\limits_{k=0}^{\infty}(k+r)a_kt^{k+r-1}$$
$$y''(t)=\sum\limits_{k=0}^{\infty}(k+r)(k+r-1)a_kt^{k+r-2}$$
Since 
$$t^2y''-t(1+t)y'+y=0$$
then we can obtain that
\begin{align*}
0=&t^2\sum\limits_{k=0}^{\infty}(k+r)(k+r-1)a_{k}t^{k+r-2}-t(1+t)\sum\limits_{k=0}^{\infty}(k+r)a_kt^{k+r-1}+\sum\limits_{k=0}^{\infty}a_kt^{k+r}\\
=&\sum\limits_{k=0}^{\infty}(k+r)(k+r-1)a_{k}t^{k+r}-\sum\limits_{k=0}^{\infty}(k+r)a_{k}t^{k+r}-\sum\limits_{k=0}^{\infty}(k+r)a_k t^{k+r+1}+\sum\limits_{k=0}^{\infty}a_k t^{k+r}\\
=&(r(r-1)a_0-ra_0+a_0)t^{r}\\
&+\sum\limits_{k=1}^{\infty}((k+r)(k+r-1)a_{k}-(k+r)a_{k}-(k+r-1)a_{k-1}+a_k)t^{k+r}
\end{align*}

So
$$\left\{
\begin{aligned}
&r(r-1)a_0-ra_0+a_0=0\\
&(k+r-1)^2a_{k}=(k+r-1)a_{k-1},k\in\mathbb{N}^*\\
\end{aligned}
\right.$$
then $r=1$ and we obtain that $\forall k\in\mathbb{N}, a_k=\dfrac{1}{k!}a_0$
$$x_1(t)=t\sum\limits_{k=0}^{\infty}a_kt^k=a_0t\sum\limits_{k=0}^{\infty}\dfrac{1}{k!}t^k=a_0te^t$$
and
\begin{align*}
x_2(t)=&\dfrac{\partial}{\partial r}\Big(t^r\sum\limits_{k=0}^{\infty}a_k(r)t^k\Big)_{r=1}=\Big(t^r\ln t \sum\limits_{k=0}^{\infty}a_k(r)t^k+t^r\sum\limits_{k=0}^{\infty}a_k'(r)t^k\Big)_{r=1}\\
=&a_0te^t\ln t+t\Big(\sum\limits_{k=0}^{\infty}a_k'(r)t^k\Big)_{r=1}
\end{align*}
Since for $k\geqslant1$,
\begin{align*}
\dfrac{a_k'(1)}{a_k(1)}=&\Big(\dfrac{d}{dr}\ln a_k(r)\Big)_{r=1}=\dfrac{d}{dr}\Big(\ln a_0(r)-\sum\limits_{i=1}^k\ln (i+r-1)\Big)_{r=1}\\
=&\dfrac{a_0'(1)}{a_0(1)}-\sum\limits_{i=1}^k\dfrac{1}{i}
\end{align*}
then for $k\geqslant1$,
$$a'_k=\Big(\dfrac{a'_0}{a_0}-\sum\limits_{i=1}^k\dfrac{1}{i}\Big)\dfrac{1}{k!}a_0$$
so
\begin{align*}
x_2(t)=&a_0te^t\ln t+t\sum\limits_{k=0}^{\infty}a_0'\dfrac{1}{k!}t^k-t(a_0\sum\limits_{k=1}^{\infty}\Big(\sum\limits_{i=1}^k\dfrac{1}{i}\Big)\dfrac{1}{k!}t^k)\\
=&a_0'te^t+a_0(te^t\ln t-\sum\limits_{k=1}^{\infty}\Big(\sum\limits_{i=1}^k\dfrac{1}{i}\Big)\dfrac{1}{k!}t^{k+1})
\end{align*}
So two independent solutions for $t^2y''-t(1+t)y'+y=0$ is
$$y_1(t)=c_1te^t,y_2(t)=c_2(te^t\ln t-\sum\limits_{k=1}^{\infty}\Big(\sum\limits_{i=1}^k\dfrac{1}{i}\Big)\dfrac{1}{k!}t^{k+1})$$

\section*{Exercise 8.4}
\subsection*{i)}
Set $y(x)=x^r\sum\limits_{k=0}^{\infty}a_kx^k$, then
$$y'(x)=\sum\limits_{k=0}^{\infty}(k+r)a_kx^{k+r-1}$$
$$y''(x)=\sum\limits_{k=0}^{\infty}(k+r)(k+r-1)a_kx^{k+r-2}$$
Since 
$$x^2y''+xy'+(x^2-\nu^2)y=0$$
then we can obtain that
\begin{align*}
0=&x^2\sum\limits_{k=0}^{\infty}(k+r)(k+r-1)a_kx^{k+r-2}+x\sum\limits_{k=0}^{\infty}(k+r)a_kx^{k+r-1}+(x^2-\nu^2)x^r\sum\limits_{k=0}^{\infty}a_kx^k\\
=&\sum\limits_{k=0}^{\infty}(k+r)(k+r-1)a_kx^{k+r}+\sum\limits_{k=0}^{\infty}(k+r)a_kx^{k+r}+\sum\limits_{k=0}^{\infty}a_kx^{k+r+2}-\nu^2\sum\limits_{k=0}^{\infty}a_kx^{k+r}\\
=&r(r-1)a_0x^r+(1+r)ra_1x^{1+r}+ra_0x^r+(1+r)a_1x^{1+r}-\nu^2a_0x^r-\nu^2a_1x^{1+r}\\
&+\sum\limits_{k=0}^{\infty}(k+r+2)(k+r+1)a_{k+2}+(k+r+2)a_{k+2}+a_k-\nu^2a_{k+2})x^{x+r+2}
\end{align*}
So
\begin{align*}
\left\{
\begin{aligned}
&(r^2-\nu^2)a_0=0\\
&((r+1)^2-\nu^2)a_1=0\\
&((k+r+2)^2-\nu^2)a_{k+2}=-a_k,k\geqslant2\\
\end{aligned}
\right.\Leftrightarrow\left\{
\begin{aligned}
&r=\nu\vee r=-\nu\\
&a_1=0\\
&((k+r+2)^2-\nu^2)a_{k+2}=-a_k,k\geqslant0\\
\end{aligned}
\right.
\end{align*}
For $r=\nu$,
\begin{align*}
a_{2k+1}&=0,k\in\mathbb{N}\\
a_{2k}=&-\dfrac{1}{(2k)(2k+2\nu)}a_{2k-2}=-\Big(\dfrac{1}{2}\Big)^2\dfrac{1}{k(k+\nu)}a_{2k-2}\\
=&\Big(\dfrac{1}{2}\Big)^{2k}\dfrac{(-1)^k}{k!\prod\limits_{i=1}^k(i+\nu)}a_0,k\in\mathbb{N}^*
\end{align*}
Set $a_0=\dfrac{1}{2^{\nu}\Gamma(\nu+1)}$, then $a_{2k}=\Big(\dfrac{1}{2}\Big)^{2k}\dfrac{(-1)^k}{k!\prod\limits_{i=1}^k(i+\nu)}\dfrac{1}{2^{\nu}\Gamma(\nu+1)}=\Big(\dfrac{1}{2}\Big)^{2k+\nu}\dfrac{(-1)^k}{k!\Gamma(1+k+\nu)}$

So one solution is
$$J_{\nu}(x)=x^{r}\sum\limits_{k=0}^{\infty}a_{k}x^k=x^{\nu}\sum\limits_{n=0}^{\infty}\Big(\dfrac{1}{2}\Big)^{2n+\nu}\dfrac{(-1)^n}{n!\Gamma(1+n+\nu)}x^{2n}=\sum\limits_{n=0}^{\infty}\dfrac{(-1)^n}{n!\Gamma(1+n+\nu)}\Big(\dfrac{x}{2}\Big)^{2n+\nu}$$

\subsection*{ii)}
For $r=-\nu$, if $2\nu$ is not an integer
\begin{align*}
\left\{
\begin{aligned}
&r=\nu\vee r=-\nu\\
&a_1=0\\
&a_{k+2}=-\dfrac{1}{(k+2)(k+2-2\nu)}a_k,k\geqslant0\\
\end{aligned}
\right.
\end{align*}
So
\begin{align*}
a_{2k+1}&=0,k\in\mathbb{N}\\
a_{2k}=&-\dfrac{1}{(2k)(2k-2\nu)}a_{2k-2}=-\Big(\dfrac{1}{2}\Big)^2\dfrac{1}{k(k-\nu)}a_{2k-2}\\
=&\Big(\dfrac{1}{2}\Big)^{2k}\dfrac{(-1)^k}{k!\prod\limits_{i=1}^k(i-\nu)}a_0,k\in\mathbb{N}^*
\end{align*}
If $0<\nu<1$, we can set $a_0=\dfrac{2^{\nu}}{\Gamma(1-\nu)}$, then $$a_{2k}=\Big(\dfrac{1}{2}\Big)^{2k}\dfrac{(-1)^k}{k!\prod\limits_{i=1}^k(i-\nu)}\dfrac{2^{\nu}}{\Gamma(1-\nu)}=\Big(\dfrac{1}{2}\Big)^{2k-\nu}\dfrac{(-1)^k}{k!\Gamma(1+k-\nu)}$$
this is also hold for $k=0$. So one solution is
$$J_{-\nu}(x)=x^{r}\sum\limits_{k=0}^{\infty}a_{k}x^k=x^{-\nu}\sum\limits_{n=0}^{\infty}\Big(\dfrac{1}{2}\Big)^{2n-\nu}\dfrac{(-1)^n}{n!\Gamma(1+n-\nu)}x^{2n}=\sum\limits_{n=0}^{\infty}\dfrac{(-1)^n}{n!\Gamma(1+n-\nu)}\Big(\dfrac{x}{2}\Big)^{2n-\nu}$$

If $\lceil\nu\rceil=m>1$,  we can set $a_0=\dfrac{2^{\nu}}{\Gamma(-\nu)}=\dfrac{2^{\nu}\prod\limits_{i=1}^{m}(i-\nu)}{\Gamma(m-\nu)}$, then $$a_{2k}=\Big(\dfrac{1}{2}\Big)^{2k}\dfrac{(-1)^k}{k!\prod\limits_{i=1}^k(i-\nu)}\dfrac{2^{\nu}\prod\limits_{i=1}^{m}(i-\nu)}{\Gamma(m-\nu)}=\Big(\dfrac{1}{2}\Big)^{2k-\nu}\dfrac{(-1)^k}{k!\Gamma(1+k-\nu)}$$
this is also hold for $k=0$. So one solution is
$$J_{-\nu}(x)=x^{r}\sum\limits_{k=0}^{\infty}a_{k}x^k=x^{-\nu}\sum\limits_{n=0}^{\infty}\Big(\dfrac{1}{2}\Big)^{2n-\nu}\dfrac{(-1)^n}{n!\Gamma(1+n-\nu)}x^{2n}=\sum\limits_{n=0}^{\infty}\dfrac{(-1)^n}{n!\Gamma(1+n-\nu)}\Big(\dfrac{x}{2}\Big)^{2n-\nu}$$
that is for $\nu<0,2\nu\notin\mathbb{N}$,
$$J_{-\nu}(x)=\sum\limits_{n=0}^{\infty}\dfrac{(-1)^n}{n!\Gamma(1+n-\nu)}\Big(\dfrac{x}{2}\Big)^{2n-\nu}$$

\subsection*{iii)}
Since $a_{2n-2}=-(2n-2+2)(2n-2+2-2n)a_{2n}=0$, $\forall 2k\leqslant 2n-2,k\in\mathbb{N},a_{2k}=0$. And $\forall k\in\mathbb{N},2k\geqslant2n+2$
\begin{align*}
a_{2k}=&-\dfrac{1}{(2k)(2k-2n)}a_{2k-2}=-\Big(\dfrac{1}{2}\Big)^2\dfrac{1}{k(k-n)}a_{2k-2}\\
=&\Big(\dfrac{1}{2}\Big)^{2(k-n)}\dfrac{(-1)^{(k-n)}}{\Big(\prod\limits_{i=n+1}^ki\Big)\Big(\prod\limits_{i=n+1}^k(i-n)\Big)}a_{2n}\\
=&\Big(\dfrac{1}{2}\Big)^{2(k-n)}\dfrac{(-1)^{(k-n)}n!}{(k-n)!k!}a_{2n}
\end{align*}
This is also true for $2k=2n$. Set $a_{2n}=\dfrac{(-1)^n}{2^{n}n!\Gamma(n)}$, then
\begin{align*}
J_{-n}(x)=&x^{r}\sum\limits_{k=0}^{\infty}a_{k}x^k=x^{-n}\sum\limits_{k=n}^{\infty}\Big(\dfrac{1}{2}\Big)^{2(k-n)}\dfrac{(-1)^{(k-n)}n!}{(k-n)!k!}a_{2n}x^{2k}\\
=&\sum\limits_{t=0}^{\infty}\Big(\dfrac{1}{2}\Big)^{2t}\dfrac{(-1)^{t}n!}{t!(t+n)!}a_{2n}x^{2t+n}\\
=&(-1)^n\sum\limits_{t=0}^{\infty}\Big(\dfrac{1}{2}\Big)^{2t+n}\dfrac{(-1)^{t}}{t!\Gamma(t+n+1)}x^{2t+n}\\
=&(-1)^nJ_n(x)
\end{align*}
So $J_{-n}=(-1)^nJ_n$ for $n\in\mathbb{N}$, and $J_{-n}$ does not yield a second independent solution.
 
\section*{iv)}
Set $y_2(x)=J_{\nu}c(x)$, since $x^2y''+xy'+(x^2-\nu^2)y=0$,
\begin{align*}
0=&x^2(J''_{\nu}(x)c(x)+2J'_{\nu}(x)c'(x)+J_{\nu}(x)c''(x))+x(J'(x)c(x)+J_{\nu}(x)c'(x))+(x^2-\nu^2)J_{\nu}(x)c(x)\\
=&c(x)(x^2J''_{\nu}(x)+xJ_{\nu}(x)+(x^2-\nu^2)J_{\nu}(x)c(x))+(2x^2J'_{\nu}(x)+xJ_{\nu}(x))c'(x)+x^2J_{\nu}(x)c''(x)\\
=&(2x^2J'_{\nu}(x)+xJ_{\nu}(x))c'(x)+x^2J_{\nu}(x)c''(x)
\end{align*}
So
\begin{align*}
&\dfrac{dc'(x)}{dx}=-2\dfrac{J'_{\nu}(x)}{J_{\nu}(x)}-\dfrac{1}{x}\dfrac{dc(x)}{dx}\\
\Rightarrow&\dfrac{dc'(x)}{dx}=(-2\dfrac{d\ln|J_{\nu}(x)|}{dx}-\dfrac{1}{x})c'(x)\\
\Rightarrow&\ln |c'(x)|=(-2\ln|J_{\nu}(x)|-\ln |x|)\\
\Rightarrow&c'(x)=\dfrac{1}{x\cdot J_{\nu}^2(x)}\\
\Rightarrow&c(x)=\int \dfrac{dx}{x\cdot J_{\nu}^2(x)}
\end{align*}
So a second solution can be formally represented as
$$y_2(x)=J_{\nu}(x)\int \dfrac{dx}{x\cdot J_{\nu}^2(x)}$$

\section*{Exercise 8.5}
\subsection*{i)}
Set $u(t)=x^{-1/2}y(x)$ and $t=\dfrac{2}{3}x^{3/2}$, then
\begin{align*}
\dfrac{du(t)}{dt}&=(\dfrac{d}{dx}x^{-1/2}y(x))\dfrac{dx}{dt}=(-\dfrac{1}{2}x^{-3/2}y(x)+x^{-1/2}y'(x))(3/2)^{2/3}\dfrac{2}{3}t^{-1/3}\\
&=(-\dfrac{1}{2}x^{-3/2}y(x)+x^{-1/2}y'(x))x^{-1/2}\\
&=-\dfrac{1}{2x^2}y(x)+\dfrac{y'(x)}{x}
\end{align*}

\begin{align*}
\dfrac{d^2u(t)}{dt^2}&=\dfrac{d}{dx}(-\dfrac{1}{2x^2}y(x)+\dfrac{y'(x)}{x})\dfrac{dx}{dt}\\
&=(\dfrac{1}{x^3}y(x)-\dfrac{1}{2x^2}y'(x)-\dfrac{y'(x)}{x^2}+\dfrac{1}{x}y''(x))x^{-1/2}\\
&=x^{-7/2}y(x)-\dfrac{3}{2}x^{-5/2}y'(x)+y''(x)x^{-3/2}
\end{align*}
So
\begin{align*}
&t^2u''+tu'+(t^2-(1/3)^2)u\\
=&\dfrac{4}{9}x^3(x^{-7/2}y(x)-\dfrac{3}{2}x^{-5/2}y'(x)+y''(x)x^{-3/2})+\dfrac{2}{3}x^{3/2}(-\dfrac{1}{2x^2}y(x)+\dfrac{y'(x)}{x})\\
&+(\dfrac{4}{9}x^3-(1/3)^2)x^{-1/2}y(x)\\
=&\dfrac{4}{9}x^{-1/2}y(x)-\dfrac{2}{3}x^{1/2}y'(x)-\dfrac{4}{9}x^{5/2}y(x)-\dfrac{1}{3}x^{-1/2}y(x)+\dfrac{2}{3}x^{1/2}y'(x)+\dfrac{4}{9}x^{5/2}y(x)-\dfrac{1}{9}x^{-1/2}y(x)\\
=&0
\end{align*}
So Airy's equation is Bessel's equation of order $\nu = 1/3$.


\subsection*{ii)}
The general solution of $t^2u''+tu'+(t^2-(1/3)^2)u=0$ is
\begin{align*}
u(t)&=c_1J_{1/3}(t)+c_2J_{-1/3}(t)
\end{align*}
So 
\begin{align*}
y(x)&=x^{1/2}u(t)=x^{1/2}(c_1J_{1/3}(t)+c_2J_{-1/3}(t))\\
&=x^{1/2}(c_1J_{1/3}(\dfrac{2}{3}x^{3/2})+c_2J_{-1/3}(\dfrac{2}{3}x^{3/2}))
\end{align*}

To sum up, the general solution to Airy's equation is
$$y(x)=x^{1/2}\Big(c_1J_{1/3}(\dfrac{2}{3}x^{3/2})+c_2J_{-1/3}(\dfrac{2}{3}x^{3/2})\Big)$$




\end{document}
